\section{Introduction}\label{sec:introduction}
Internet topology can be defined as the structure in which various internet components such as ISPs, routers, switches and end systems are connected to each other\cite{encyclopedia1}. In the last decade, internet connectivity has increased as many devices have been connected to the internet which in turn has led to rapid change of the internet topology. Oftenly, internet topology is a map where the nodes represent the various internet components and the links represent the interconnections between those components. To facilitate the topology mapping, topology discovery techniques are used to collect internet measurements from different internet measuring platforms\cite{Donnet2007}. These topology techniques are divided into two: passive techniques and active techniques.

Passive techniques use the BGP routing table to infer topology data that is used to map internet topology on AS-level and router level. Active techniques involve sending internet traffic to selected network destinations with an aim to collect topological data. The responses received from the destinations are then sampled and analysed to determine the routing path and the Round Trip Time(RTT) of the traffic\cite{Donnet2007}. 

Internet Exchange Points(IXPs) can be described as locations where ISPs and CDNs(Content Delivery Networks) meet and exchange internet traffic\cite{effectsofIXPS}. IXPs are important as they facilitate peering amongst ISPs\cite{effectsofIXPS}. Peering is a business relationship between ISPs where the ISPs agree to exchange internet traffic from each other. Peering reduces end to end traffic’s latency and improves the internet’s quality of service. 

According to AFRINIC\cite{Africanp74:online}, the African Regional Internet Registry, Africa had 46 IXPs spread across 34 countries by August 2020. Most of these IXPs remain underutilised as according to PeeringDB\cite{PeeringD7:online}, more than 60\% of these IXPs have less than 10 ISPs peering per IXP.  Most African ISPs have been found to peer at IXPs outside the continent. This has in turn affected the intra-continental traffic negatively as it experiences high end to end latency since the traffic is forced to be routed outside the continent so as to get to its destination \cite{africanet1}. According to Chavula \textit{et al} \cite{Africa2}, internet performance in Africa can be improved if peering of African ISPs is increased at African IXPs. To determine how best this peering can be done, various peering and routing  scenarios need to be simulated and tested using network simulators. 

\subsection{Project Objectives}
The objective of this project was to develop and create an online internet topology simulation platform that would evaluate how various ISPs could peer at Africa IXPs and how the internet performance affected would be affected. 

This paper describes the agile process of designing and creating the simulator and how the simulator was used to simulate the internet topology under various routing and peering conditions. The results from various scenarios are also described and discussed in the rest of the paper.  
