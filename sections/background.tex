
\section{BACKGROUND}
 To simulate the mesh network of the internet, various network simulation techniques are used. These simulation techniques differs from each other depending on the nature of the simulator, the conditions under which the simulator runs on and also the type of initial topology that is used during simulation. Depending on the nature of the simulator, network simulation techniques can be described to be running on a web-based simulator or on a non-web simulator. Web based simulators are easier to use compared to non-web simulators as they don't require prior installation. Network simulation may use a synthetic network topology or non-synthetic topology. Synthetic topologies are generated from modelling the network from mathematical algorithms but non-synthetic topologies are generated from real measurements taken from the network. Due to the ever-changing nature of internet, it is not effective to use synthetic topologies to simulate the internet. It is important to run Internet measurements and use the data collected to generate the initial topology. Simulation results obtained from simulating non-synthetic topologies are more reliable compared to r5results obtained from simulating synthetic topologies.  
 
 Simulation of internet graphs involves adjusting the number of nodes or adjusting the number of edges \cite{Internetgraph} present on the graph. The adjustment can be done by removing or adding either nodes or the edges on the graph. The aim of such simulation is to test how the internet performance can be affected when different network conditions are introduced to internet topology. When a network condition is introduced, the topology is said to be under a certain scenario. To simulate the topology, the mapped topology is subjected to the condition which in turn changes the routing of the mapped nodes. Internet traffic is then sent from source to a selected destination and the link delay between the source and destination is measured and recorded. Internet topology mapped at AS-level may be simulated under the following conditions: adding an IXP node, adding links between nodes, removing links between nodes and removing nodes.  

\textbf{Adding an IXP node followed by adding links between nodes and the introduced IXP node:} This condition tests what could happen if an IXP is introduced at a certain location and some ISPs decide to connect through it.  During simulation, when an IXP node is added to the internet topology, selected nodes are then connected to the IXP. The node degree of the connected nodes increases and if some nodes are near the IXP, the link delay of traffic between these nodes is expected to decrease. The results obtained from this scenario are used to show what could be the effect of introducing an IXP node at a certain location. 

\textbf{Removing a link between two nodes:} This condition tests what could happen to the internet traffic flow if a link between two ISPs  is removed or goes down. During simulation, when a link between two nodes is removed, traffic flowing between the two nodes is expected to look for an alternative route. The link delay of the alternative route is then measured to indicate how the internet performance was affected by the link removal. 

\textbf{Removing node:} This condition tests what happens if one of the ISPs closes or ceases operations. During simulation, if a node is removed, its associated links are also removed. If any other nodes were connecting through the removed node, the internet traffic flow is expected to change as routing has been affected.  


\section{Related Work}\label{sec:Related Work}
%Are there any theories, concepts, terms, and ideas that may be unfamiliar to the target audience and will require you to provide any additional explanation?
%Any historical data that need to be shared in order to provide context on why the current issue emerged?
%Are there any concepts that may have been borrowed from other disciplines that may be unfamiliar to the reader and need an explanation?
Fanou \textit{et al} \cite{Africa3} used 214 RIPE Atlas probes and took active measurements from 32 African countries, covering 90 ASes present in these countries. The results indicated that it was important to have a large and diversified set of vantage points before drawing conclusions on the state of interdomain routing in Africa which is basically continent Internet topology. This is because transit habits of various ISPs in Africa depend on the official language of the country, monetary region and business profile of the region. For example; Orange was found to be the most dominant ISP in West Africa countries which mostly speak French and it was present in France IXP. Orange was not found to be present in English speaking countries. These results clearly show that to make conclusions on the state of Africa’s internet topology, internet measurements need to be taken from a broad range of vantage points \cite{Africa3}.

Chavula \textit{et al} \cite{Africa2}  took topology measurements from 5 vantage points targeting 95 academic institutions in Africa. These measurements were used to generate topology maps at both AS-level and Point of Presence(PoP) level. It was found that 75\% of the internet traffic from African vantage points to academic institutions, traversed through PoPs in Europe such as Amsterdam, London,Lisbon and Marseille. The observation of the topology at AS-level, showed that many NRENs were interconnected mostly by ISPs that peered at global IXPs in Europe [9]. Due to this, the node degree of the African internet topology at AS-level was high for the European ASes compared to African ASes \cite{Africa2}.

Chavula \textit{et al} \cite{simulation11} carried out topology simulation where links connecting local ISPs with global IXPs mostly Europe were transferred to local Africa’s IXPs. A proxy Africa Internet Exchange Point (IXP) was also created for the simulation purposes. The results showed that by utilising local IXPs, end to end latency for intra-continent traffic reduced by 50\% \cite{simulation11}. The simulation was also done using the Software Defined IXPs to try and see the effect of introducing traffic engineering principles in an IXP settings. 

Most of the existing network simulators use synthetic topologies to carry out their simulation. The simulators are divide into two: the web based simulators and non-web based simulators. Non-web based simulators includes: ns-family simulators\cite{ns3} and Fast Network Simulation Setup tool chain(FNSS) \cite{simulation2}. IXP Jedi tool\cite{JEDI:online} that has been developed RIPE Atlas is one of the web based network topology simulator. Due to the ever-changing structure of internet topology \cite{simulation1}, using synthetic topologies to carry out internet simulation may not provide reliable results. This problem have influenced our project to develop a simulation interface that simulates an internet topology generated from real data. Web based simulators are effective such that they can be accessed without prior installation while non-web based simulators require installation which can be tedious and cumbersome to the users. Hence, in this project we decided to create a web based simulator that can be accessed anytime without requiring prior installation. 

In pursuit of making data on current internet traffic statistics, Africa
Route Data Analyzer(ARDA) \cite{AfricanR17:online,fanou2019system}, was created. ARDA is an open source web platform created by AFRINIC to provide a common IXP data collection in Africa \cite{AfricanR17:online}. A user can be able to see the traffic statics at three different views: IXP-view, National-view and regional
view. 



